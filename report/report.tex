\title{Android Mini-App: Quiz Player}
\author{
        Richard Dizon \& Emma Kopf\\
        CS 4720: Team Magneton\\
        University of Virginia\\
}
\date{September 19, 2016}

\documentclass[12pt]{article}

\begin{document}
\maketitle

\section{Instructions and Information}
Given \(T(n)=2 \cdot T(\frac{n}{2})+n^2\) where \(a=2\) and \(b=2\), we can find that \(k=log_2 2=1\). This fulfills case 3 of the Master theorem where \(f(n)\in \Omega (n^{k+\epsilon } )\) for a positive \(\epsilon , n^2 \in n^{1+1} \) which leads us to the conclusion that \(T(n) \in \theta(n^2)\) \newline

\section{Lessons Learned}
This is similar to the balance puzzle where we must divide the list into three, close-lengthed parts. We call \(f()\) on two of those three parts and if both of those parts have equal length and the function returns zero, then we know for sure that the two does not exist in either of those two segments. If one side is "heavier" then we know that the two must exist in that side. \newline

\noindent From there, we perform the same operation on either the third part or the heavier side by splitting it as well into three parts and doing a comparison until we can isolate three single indices. Ultimately, either one side will return a non-zero value to identify the two or it will return zero in turn identifying that the remaining index is the two. \newline

\noindent Given that the divide section does not result in three equally lengthed segments, one segment may have a length that is one longer than another. We then expect that side to have a greater sum, but if for some reason we discover that they have the same sum, then the shorter segment must contain the two and the pattern continues. In the ideal length \(n\), this results in a runtime of exactly \(O(log_3n)\) but this and all other cases still simplifie down to a runtime of \(O(logn)\)

\end{document}
